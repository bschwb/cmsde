\documentclass[a4paper,11pt]{scrartcl}
\usepackage[utf8]{inputenc}
\usepackage[english]{babel}

\usepackage[headsepline]{scrlayer-scrpage}
\ihead{Bernd Schwarzenbacher}
\chead{CMSDE HW4}
\ohead{\today}

\usepackage{amsmath}
\usepackage{amssymb}
\usepackage{commath}
\usepackage{mathtools}
\usepackage[retainorgcmds]{IEEEtrantools}

\usepackage{hyperref}
\usepackage[noabbrev]{cleveref}
\usepackage{graphicx}
\usepackage{listings}
\lstset{language=Python,
  frame=single,
  breaklines=true,
  captionpos=b
}

\newcommand*{\R}{\mathbb{R}}
\newcommand*{\EV}[1]{\mathbb{E}\left[{#1}\right]}
\newcommand*{\Dt}{\Delta{}t}
\newcommand*{\DW}{\Delta{}W}
\newcommand*{\fdt}{\frac{\partial{}f}{\partial{}\theta}}

\usepackage{enumitem}

\begin{document}

\begin{enumerate}

\item

\begin{enumerate}[leftmargin=1em]
  \item
    \[f(\theta,y) \coloneqq (\theta - y)^2 \Rightarrow
      \fdt = 2 (\theta - y) \]

    I prove by induction:
    \begin{itemize}
      \item \[\EV{\theta_0^2} = 1 \leq C\]

      \item
        \begin{IEEEeqnarray*}{rCl}
          \EV{\theta_{n+1}^2} &=& \EV{\left(\theta_n - \Dt
              \fdt(\theta_n, Y_n)\right)^2}
          = \EV{\left(\theta_n-2\Dt(\theta_n-Y_n)\right)^2}\\
          &=& \EV{\theta_{n}^2} - 4 \Dt \EV{\theta_n(\theta_n-Y_n)} + 4 \Dt^2
          \EV{(\theta_n-Y_n)^2} \\
          &=& \EV{\theta_n^2} - 4\Dt\EV{\theta_n^2} + 4 \Dt \EV{\theta_n Y_n} +
          4\Dt^2\EV{\theta_n^2} - 8\Dt^2\EV{\theta_n Y_n} + 4 \Dt^2 \EV{Y_n^2} \\
          &=& \EV{\theta_n^2} (1 - 4\Dt + 4\Dt^2) + 4\Dt^2 \leq C ( 1-4\Dt+4\Dt^2) +
          4 \Dt^2
        \end{IEEEeqnarray*}

        The last expression should be smaller than $C$:
        \begin{IEEEeqnarray*}{rCl}
          C(1 - 4\Dt + 4\Dt^2) + 4\Dt^2 &\leq& C \\
          C(-4\Dt + 4\Dt^2) + 4\Dt^2 &\leq& 0 \\
          \frac{4\Dt^2}{4\Dt - 4\Dt^2} &\leq& C \\
          \frac{\Dt}{1 - \Dt} &\leq& C
        \end{IEEEeqnarray*}
    \end{itemize}

    \[\EV{f(\theta,Y)} = \EV{\left(\theta-Y\right)^2} = \EV{\theta^2} -
      2\EV{\theta Y} + \EV{Y^2} = \theta^2 + 1\]
    is smallest for $\theta = 0$.

    No convergence? Looking at the plot, we see some convergence rate, but then
    oscillating within some bound around 0.

    \item
      \begin{figure}[h]
        \centering
        \includegraphics[width=\linewidth]{pic/theta_conv.pdf}
        \caption{$\theta$ convergence}
        \label{fig:theta_conv}
      \end{figure}

    \item
      Ornstein-Uhlenbeck process
      \[ \dif{}X_t = \alpha(\mu-X_t)\dif{}s+\sigma\dif{}W \]
      Forward Euler:
      \[ X_{n+1} - X_n = \alpha(\mu - X_n) \Dt + \sigma \DW_n \]

      \[\theta_{n+1} - \theta_n = 2 \Dt \theta_n - 2 \sqrt{\Dt} \DW_n)\]

      Since $\DW_n = \sqrt{\Dt} Y_n$
      $\alpha = 2, \mu = 0, \sigma = 2 \sqrt{\Dt}$

      Is it okay, that $\Dt$ appears in the definition of $\sigma$?
      It seems weird, because then the forward Euler method approximates a
      different process for a different $\Dt$.

      Gradient descent:
      \begin{IEEEeqnarray*}{rCl}
      \theta_{n+1} &=& \theta_n - \Dt \frac{\partial}{\partial{}\theta} \EV{(\theta_n - Y)^2}
        = \theta_n - \Dt \frac{\partial}{\partial{}\theta}\left( \EV{\theta_n^2}
        - 2 \EV{\theta_n Y} + \EV{Y^2}\right) \\
      &=& \theta_n - \Dt \frac{\partial}{\partial{}\theta}\left(\theta_n^2 + 1\right)
      = \theta_n - 2\Dt \theta_n
      \end{IEEEeqnarray*}

      Convergence rate:
      \begin{IEEEeqnarray*}{rCl}
      \EV{\theta_{n}^2} &=& \EV{\theta_{n-1}^2} - 4\Dt\EV{\theta_{n-1}^2} +
      4\Dt^2\EV{\theta_{n-1}^2} \\
      &=& \EV{\theta_{n-1}^2} (1 - 4\Dt + 4\Dt^2) \\
      &=& \EV{\theta_0^2} (1 - 4\Dt + 4\Dt^2)^n
      \underset{n\rightarrow\infty}{\longrightarrow} 0
      \end{IEEEeqnarray*}

      So convergence rate is $O\left((1-\Dt)^n\right)$?
\end{enumerate}

\item

\begin{enumerate}[leftmargin=1em]
  \item
    \begin{figure}[h]
        \begin{minipage}[b]{.5\linewidth}
          \centering
          \includegraphics[width=\linewidth]{pic/learned_fun_5e-02.pdf}
          \caption{Learned function $\Dt=0.05$}
          \label{fig:learned_fun_0.05}
        \end{minipage}%
        \begin{minipage}[b]{.5\linewidth}
          \centering
          \includegraphics[width=\linewidth]{pic/test_error_5e-02.pdf}
          \caption{Test error $\Dt=0.05$}
          \label{fig:test_error_0.05}
        \end{minipage}
    \end{figure}
    \begin{figure}[h]
        \begin{minipage}[b]{.5\linewidth}
          \centering
          \includegraphics[width=\linewidth]{pic/learned_fun_5e-03.pdf}
          \caption{Learned function $\Dt=0.005$}
          \label{fig:learned_fun_0.005}
        \end{minipage}%
        \begin{minipage}[b]{.5\linewidth}
          \centering
          \includegraphics[width=\linewidth]{pic/test_error_5e-03.pdf}
          \caption{Test error $\Dt=0.005$}
          \label{fig:test_error_0.005}
        \end{minipage}
    \end{figure}
    \begin{figure}[h]
        \begin{minipage}[b]{.5\linewidth}
          \centering
          \includegraphics[width=\linewidth]{pic/learned_fun_5e-04.pdf}
          \caption{Learned function $\Dt=0.0005$}
          \label{fig:learned_fun_0.0005}
        \end{minipage}%
        \begin{minipage}[b]{.5\linewidth}
          \centering
          \includegraphics[width=\linewidth]{pic/test_error_5e-04.pdf}
          \caption{Test error $\Dt=0.0005$}
          \label{fig:test_error_0.0005}
        \end{minipage}
    \end{figure}

  \item
    \begin{figure}[h]
      \centering
      \includegraphics[width=\linewidth]{pic/exp_vs_emp_loss.pdf}
      \caption{Empirical loss function $E_a$ and expected loss function $E_1$
        for $\Dt = 0.005$}
      \label{fig:empirical_loss}
    \end{figure}

  \item
    There is fast convergence for $K=1$ (see \cref{fig:test_error_K1}) since we
    only have a few parameters to train.
    The function fit is bad, since there are only parameters to fit a single
    sigmoid function.

    \begin{figure}[h]
        \begin{minipage}[b]{.5\linewidth}
          \centering
          \includegraphics[width=\linewidth]{pic/learned_fun_K1.pdf}
          \caption{Learned function $\Dt=0.005, K=1$}
          \label{fig:learned_fun_K1}
        \end{minipage}%
        \begin{minipage}[b]{.5\linewidth}
          \centering
          \includegraphics[width=\linewidth]{pic/test_error_K1.pdf}
          \caption{Test error $\Dt=0.005, K=1$}
          \label{fig:test_error_K1}
        \end{minipage}
    \end{figure}

   \item
     In \cref{fig:test_error_K1} we observe, that we need around 10 samples for
     the single node.
     Assuming this scales linearly, we need in the order of 10 samples per node.
     So the 100 samples for 10 nodes seem to be the right order of magnitude.

   \item
     To obtain a small value for the empirical cost function, but a large value
     for the expected cost function, the model needs to overfit.
     This means to make the model highly dependent on the training data, but bad
     to generalize.
     We can achieve this with a low number of computations $M$ for a high number
     of nodes $K$.

   \item
    \begin{figure}[h]
        \begin{minipage}[b]{.5\linewidth}
          \centering
          \includegraphics[width=\linewidth]{pic/learned_fun_bias.pdf}
          \caption{Learned function $\Dt=0.005$ with bias}
          \label{fig:learned_fun_bias}
        \end{minipage}%
        \begin{minipage}[b]{.5\linewidth}
          \centering
          \includegraphics[width=\linewidth]{pic/test_error_bias.pdf}
          \caption{Test error $\Dt=0.005$ with bias}
          \label{fig:test_error_bias}
        \end{minipage}
    \end{figure}

    \begin{figure}[h]
        \begin{minipage}[b]{.5\linewidth}
          \centering
          \includegraphics[width=\linewidth]{pic/learned_fun_sin.pdf}
          \caption{Learned function $\Dt=0.005$ with sin activation function}
          \label{fig:learned_fun_sin}
        \end{minipage}%
        \begin{minipage}[b]{.5\linewidth}
          \centering
          \includegraphics[width=\linewidth]{pic/test_error_sin.pdf}
          \caption{Test error $\Dt=0.005$ with sin activation function}
          \label{fig:test_error_sin}
        \end{minipage}
    \end{figure}

    \item
     \[\min_{\theta\in\Theta} \EV{\left( \alpha_\theta(X) - f(X) \right)^2}\]
     \[f(x) = \left(x-\frac{1}{2}\right)^T\left(x-\frac{1}{2}\right), \quad
      x \in G \subset \R^d\]
      where $X \sim \mathcal{U}(G)$.
      \[\alpha_\theta(x) = \sum^K_{k=1}\phi_k^1\sigma\left( x^T \theta^2_k +
          \theta^3_k \right), \quad x \in G, \quad
        \theta = \left(\theta^1_k, \theta^2_k, \theta^3_k\right) \in \Theta = \R^K \times
        \R^{d\times K} \times \R^K\]
      with some activation function $\sigma(x)$.

    \begin{figure}[h]
      \centering
      \includegraphics[width=\linewidth]{pic/test_error_3dim.pdf}
      \caption{Test error 3-dimensional}
      \label{fig:multidim}
    \end{figure}
\end{enumerate}

\end{enumerate}
 
\section*{Code Appendix}

All code can be found online at
\url{https://github.com/bschwb/cmsde/tree/master/hw4}

\lstset{caption={Code for 1b}, label=lstex1b}
\lstinputlisting{exercise1b.py}

\lstset{caption={Code for 2g}, label=lstex2g}
\lstinputlisting{exercise2g.py}

\end{document}